\documentclass[11pt,a4paper,oneside]{report}
%\VignetteIndexEntry{Tutorial}
\usepackage{authblk}
\usepackage{hyperref}
\usepackage[utf8]{inputenc}
\usepackage{Sweave}
\begin{document}
\title{Passo a passo de como estimar elasticidades de grupos de receita usando
  o pacote ElastH}
\author[1]{Caio Figueiredo}
\affil[1]{Secretaria de Política Econômica do Ministério da Fazenda do Brasil} 
\maketitle
\tableofcontents


\section*{Introdução}

O pacote ElastH foi desenvolvido com a intenção de prover uma
maneira simples de replicar as estimações de elasticidades de grupos de
receitas, de interesse para a metodologia de cálculo do resultado estrutural da SPE/MF.
\footnote{\url{http://www.spe.fazenda.gov.br/assuntos/politica-fiscal-e-tributaria/resultado-fiscal-estrutural}}

Este tutorial parte do pressuposto de que o leitor já tem conhecimento 
básico sobre o funcionamento do software estatístico R, e que consiga importar e
exportar dados, criar variáveis e trabalhar com séries de tempo.

\section*{Estrutura do Modelo Linear Dinâmico}

Antes de prosseguir para as funções deste pacote é importante explicar e
estrutura do modelo linear dinâmico usado para estimar os componentes não
observados, incluindo a elasticidade:

\[y_t = \mu_t + \beta_t \cdot{X_t} + \gamma_t + \varepsilon_t\]
\[\mu_t = \mu_{t-1} + \nu_{t-1} + \xi_t\]
\[\nu_t = \nu_{t-1} + \zeta_t\]
\[\gamma_t = \gamma_{1,t} + \gamma_{2,}\]
\[\gamma_{1,t} = - \gamma_{1,t-2} + \omega_{1,t}\]
\[\gamma_{2,t} = - \gamma_{2,t-1} + \omega_{2,t}\]
\[\beta_t = \beta_{t-1} + \eta_t\]

Onde \(y_t\) é a série de tempo a ser decomposta, no caso de
interesse deste pacote, essa variável costuma ser um grupo de receita. \(X_t\)
deve ser uma série de tempo com uma ou mais variáveis explicativas. No caso de
interesse, essa variável costuma ser o Hiato do Produto ou o Hiato do Petróleo.

\(\mu_t\), \(\nu_t\), \(\gamma_t\) e \(\beta_t\) são os componentes não
observados estimados pelo Filtro de Kalman, respectivamente, nível,
inclinação, sazonalidade e coeficiente(s)

Os resíduos destas equações seguem as seguintes distribuições: 

\[\varepsilon_t \sim \mathcal{N}(0, \sigma^2_\varepsilon)\]
\[\xi_t \sim \mathcal{N}(0, \sigma^2_\xi)\]
\[\zeta_t \sim \mathcal{N}(0, \sigma^2_\zeta)\]
\[\omega_{1,t} \sim \mathcal{N}(0, 2\sigma^2_\omega)\]
\[\omega_{2,t} \sim \mathcal{N}(0, \sigma^2_\omega)\]
\[\eta_t \sim \mathcal{N}(0, \sigma^2_\eta)\]

Algumas hipóteses podem ser feitas sobre o comportamento dos componentes, que
podem ser estocásticos (padrão), fixos ou ignorados. Um componente estocástico
segue a estrutura de um passeio aleatório (com drift no caso do nível); um
componente fixo apresentará o mesmo valor em todos os períodos; já um componente ignorado
é excluído, junto com seus efeitos, das equações definidas acima.

Este pacote, então, se propõe a estimar os componentes sob diferentes hipóteses
e prover as ferramentas de testes que permitam comparar os modelos, subsidiando a
escolha da melhor forma funcional sob a qual as elasticidades desvem ser estimadas. 

\section*{Instalação e Carregamento}

Antes de prosseguir é necessário, caso ainda não tenha sido feito, instalar e
carregar a pacote de estimação de elasticidades da SPE/MF. Para isso, digite os
seguintes comandos no console do R:

\begin{Schunk}
\begin{Sinput}
> install.packages("ElastH")
\end{Sinput}
\end{Schunk}
\begin{Schunk}
\begin{Sinput}
> library(ElastH)
\end{Sinput}
\end{Schunk}

\section*{Funções do ElastH}

O primeiro passo para estimar as elasticidades é importar as séries de
receitas para o R. Para fins deste tutorial, será usada uma série de exemplo
disponibilizada neste pacote. Para ter acesso a ela, basta digitar o seguinte
comando no console do R:

\begin{Schunk}
\begin{Sinput}
> data(Exemplo)
\end{Sinput}
\end{Schunk}

Assim, a variável \texttt{Exemplo} estará disponível. Esta variável
consiste em uma lista de dados de interesse, incluindo séries de tempo
similares (mas não iguais) aos grupos de receitas de interesse. Para
visualizá-las, digite: 

\begin{Schunk}
\begin{Sinput}
> Exemplo$y
\end{Sinput}
\begin{Soutput}
         Qtr1     Qtr2     Qtr3     Qtr4
2005 8.468764 8.456690 8.599470 8.594502
2006 8.589788 8.697490 8.809763 8.834902
2007 8.799565 8.886277 9.074360 9.134457
2008 9.199340 9.299899 9.485294 9.471351
2009 8.934018 8.804211 9.028303 9.219753
2010 9.271527 9.317269 9.441083 9.486182
2011 9.447879 9.526106 9.651847 9.654834
2012 9.482889 9.560783 9.490462 9.511258
2013 9.382203 9.497319 9.541160 9.354080
2014 9.260405 9.183220 9.229239 9.206761
\end{Soutput}
\begin{Sinput}
> Exemplo$Hpib
\end{Sinput}
\begin{Soutput}
              Qtr1          Qtr2          Qtr3          Qtr4
1997 -0.0021561715  0.0072766419 -0.0097881772 -0.0064660835
1998 -0.0161704444 -0.0149841336 -0.0034022772 -0.0046802549
1999  0.0097203565  0.0080387997  0.0048600295  0.0017400091
2000  0.0535383046  0.0033112412  0.0727705217 -0.0403442535
2001 -0.0133474694 -0.0139414899 -0.0157975711  0.0030958321
2002  0.0011423274  0.0073170240  0.0120591076  0.0134010693
2003 -0.0106558455  0.0118463839  0.0089797693  0.0089602312
2004  0.0065060722  0.0024627787  0.0283609871 -0.0051798943
2005 -0.0021561715  0.0072766419 -0.0097881772 -0.0064660835
2006 -0.0161704444 -0.0149841336 -0.0034022772 -0.0046802549
2007  0.0097203565  0.0080387997  0.0048600295  0.0017400091
2008  0.0535383046  0.0033112412  0.0727705217 -0.0403442535
 [ reached getOption("max.print") -- omitted 7 rows ]
\end{Soutput}
\end{Schunk}

Para estimar os componentes não observados, basta utilizar o comando
\texttt{criar.dlm}. Em sua forma mais simples, é necessário apenas informar a
série de tempo a ser usada como variável a ser decomposta:

\begin{Schunk}
\begin{Sinput}
> modelo <- criar.dlm(Exemplo$y)
\end{Sinput}
\end{Schunk}

É importante destacar que, para funcionar corretamente, a
variável a ser decomposta seja uma série de tempo, ou seja de classe "ts":

\begin{Schunk}
\begin{Sinput}
> class(Exemplo$y)
\end{Sinput}
\begin{Soutput}
[1] "ts"
\end{Soutput}
\end{Schunk}

A função retorna uma longa lista de variáveis, de classe "mee", e que agora
está salva sob o nome de \texttt{modelo}. Uma das variáveis desta lista é a
\texttt{comp}, que exibe o valor estimado para todos os componentes. Para
visualizá-la, digite:

\begin{Schunk}
\begin{Sinput}
> modelo$comp
\end{Sinput}
\begin{Soutput}
NULL
\end{Soutput}
\end{Schunk}

Entre as diversas outras variáveis retornadas (vide \texttt{str(modelo)} para
uma lista completa), destacam-se: os testes de independência de resíduos
(variável \texttt{q}), o teste de homocedasticidade (variável \texttt{h}) e
teste de normalidade dos resíduos (variável \texttt{nt}). Para visualizá-los,
utilize:

\begin{Schunk}
\begin{Sinput}
> modelo$q
\end{Sinput}
\begin{Soutput}
           Q.valor Q.critico    pvalor
X-squared 5.722657  14.06714 0.5724805
\end{Soutput}
\begin{Sinput}
> modelo$h
\end{Sinput}
\begin{Soutput}
   H.valor H.critico     pvalor
1 6.976677   2.81793 0.00159846
\end{Soutput}
\begin{Sinput}
> modelo$nt
\end{Sinput}
\begin{Soutput}
  Normality.test Normality.critico    pvalor
1       2.365018          5.991465 0.3065087
\end{Soutput}
\end{Schunk}

Os testes são sempre comparados a seus valores críticos. O
modelo passa no teste se o valor for menor que seus valores críticos. Para mais
informações sobre os resultados retornados por \texttt{criar.dlm}, leia a página
de ajuda dessa função com: \texttt{help(criar.dlm)}.

\subsection*{Inclusão de variáveis independentes}

Para incluir variáveis explicativas, basta definir o argumento \texttt{X}:

\begin{Schunk}
\begin{Sinput}
> modelo <- criar.dlm(Exemplo$y, X=Exemplo$Hpib)
> modelo$comp
\end{Sinput}
\begin{Soutput}
NULL
\end{Soutput}
\end{Schunk}

Note que agora temos um componente a mais: o coeficiente da variável
explicativa, que, no caso particular deste pacote, representa a elasticidade de
um grupo de receita ante o Hiato do PIB e é aproximadamente igual a \texttt{0.40}.

\subsection*{Controle de comportamento dos componentes}

Para controlar as hipóteses sobre os componentes, vide capitulo 2, utilizamos
os argumentos \texttt{nivel}, \texttt{inclinacao}, \texttt{sazon} e \texttt{regres},
que controlam, respectivamente, os componentes de: nível, inclinação,
sazonalidade e coeficientes. Os argumentos podem ser definidos como "S", para
componentes estocásticos (padrão), "F" para componentes fixos e "N" para
componentes ignorados. Deste modo, para ignorar a sazonalidade e
fixar a inclinação, mantendo os outros parâmetros estocásticos, digite:

\begin{Schunk}
\begin{Sinput}
> modelo <- criar.dlm(Exemplo$y, X=Exemplo$Hpib, sazon="N", inclinacao="F")